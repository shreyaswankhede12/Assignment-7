% Welcome to Overleaf --- just edit your LaTeX on the left,
% and we'll compile it for you on the right. If you open the
% 'Share' menu, you can invite other users to edit at the same
% time. See www.overleaf.com/learn for more info. Enjoy!
%
%%%%%%%%%%%%%%%%%%%%%%%%%%%%%%%%%%%%%%%%%%%%%%%%%%%%%%%%%%%%%%%


% Inbuilt themes in beamer
\documentclass{beamer}

% Theme choice:
\usetheme{CambridgeUS}

% Title page details: 
\title{Assignment 7\\Probability and Random Variables} 
\author{Shreyas Wankhede}
\date{\today}
\institute{IIT Hyderabad}
\logo{\large \LaTeX{}}


\begin{document}

% Title page frame
\begin{frame}
    \titlepage 
\end{frame}

% Remove logo from the next slides
\logo{}


% Outline frame
\begin{frame}{Outline}
    \tableofcontents
\end{frame}


% Lists frame
\section{Question}
\begin{frame}{Question}
\begin{block}{question}
  Suppose z has an F distribution with (m, n) degrees of freedom.\\ (a) Show that $\dfrac{1}{z}$ also has
an F distribution with (n, m) degrees of freedom.\\ (b) Show that mz/(mz + n) has a beta
distribution
\end{block}


\end{frame}


% Blocks frame
\section{solution part a}
\begin{frame}{solution part a}
   $Z\sim F(m,n)$ Let,
   \begin{align}
   Y=\dfrac{1}{Z}\nonumber
   \end{align}
   Then ,
   \begin{align}
   F_Y(y)&=\dfrac{1}{dy/dz}f_z(1/y)\nonumber\\\vspace{5mm}
  &=\dfrac{1}{y^2}\dfrac{(m/n)^{m/2}}{\beta(n/2,m/2)}\dfrac{1}{y^{m/2 -1}}\dfrac{1}{(1+m/ny)^{m+n/2}}\nonumber\\\vspace{5mm}
  &=\dfrac{(n/m)^{n/2}}{\beta(n/2,m/2)}y^{n/2 -1}(1+\dfrac{n}{my})^{-(m+n)/2}\nonumber\\\vspace{5mm}
  &\sim F(n,m)\nonumber
   \end{align}
\end{frame} 

\section{solution part b}
\begin{frame}{solution part b}
   \begin{align}
   W&=\dfrac{Zm}{Zm+n}\nonumber\\\nonumber\\
   F_W(w)&=P(W\le w)=P(\dfrac{Zm}{Zm+n}\le w)\nonumber\\\nonumber\\
   &=P(Z \le \dfrac{nw}{m(1-w)})= F_z(\dfrac{nw}{m(1-w)})\nonumber
   \end{align}
\end{frame}

\begin{frame}   
   which gives,
   \begin{align}
   f_W(w)&=\dfrac{n}{m(1-w)^2}f_z(\dfrac{nw}{m(1-w)})\nonumber\\\nonumber\\
   &=\dfrac{n}{m(1-w)^2}\dfrac{(m/n)^{m/2}}{\beta(m/2,n/2)}(\dfrac{nw}{m(1-w)})^{m/2 -1}(1+\dfrac{w}{(1-w)})^{-(m+n)/2}\nonumber\\\nonumber\\
   &=\dfrac{1}{\beta(m/2,n/2)}w^{m/2 -1}, 0<w<1 \nonumber
   \end{align}
   Thus W has beta distribution.
   
\end{frame}



\end{document}